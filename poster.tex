% Gemini theme
% https://github.com/anishathalye/gemini

\documentclass[final]{beamer}

% ====================
% Packages
% ====================

\usepackage[T1]{fontenc}
\usepackage{lmodern}
\usepackage[orientation=portrait,size=a0,scale=1.0]{beamerposter}
\usetheme{gemini}
\usecolortheme{gemini}
\usepackage{graphicx}
\usepackage{svg}
\usepackage{booktabs}
\usepackage{tikz}
\usepackage{pgfplots}
\usepackage{pifont}
\usepackage{tcolorbox}
\usepackage{booktabs}
\usepackage{xcolor}
\usepackage{forest}
\usepackage{wrapfig}
\usepackage{multicol}
\usepackage{natbib}
\usepackage{tcolorbox}
\usepackage{hyperref}

\renewcommand{\bibpreamble}{\setlength{\columnsep}{30pt}\begin{multicols}{3}}
\renewcommand{\bibpostamble}{\end{multicols}}

\usetikzlibrary{calc,fit,shapes.arrows,positioning}
\usetikzlibrary{graphs}
\usetikzlibrary{graphdrawing}
\usetikzlibrary{trees,matrix}
\usegdlibrary{trees}
% ====================
% Lengths
% ====================

% If you have N columns, choose \sepwidth and \colwidth such that
% (N+1)*\sepwidth + N*\colwidth = \paperwidth
\newlength{\sepwidth}
\newlength{\colwidth}
\setlength{\sepwidth}{0.0333\paperwidth}
\setlength{\colwidth}{0.4\paperwidth}

\newcommand{\separatorcolumn}{\begin{column}{\sepwidth}\end{column}}
\pgfplotsset{compat=1.16}

\colorlet{darkyellow}{yellow!60!black}
\colorlet{lightblue}{blue!65}
\colorlet{darkred}{red!60!black}

\definecolor{mycolor}{rgb}{0.122, 0.435, 0.698}

\newtcbox{\mybox}{on line,
  colframe=mycolor,colback=mycolor!10!white,arc=4pt,boxsep=0pt,left=6pt,right=6pt,boxrule=0pt,top=6pt,bottom=6pt}

% \renewcommand{\emph}[1]{\mybox{\emph{#1}}}
\newcommand{\realemph}[1]{\mybox{\textit{#1}}}

% ====================
% Title
% ====================

\newcommand\textbox[1]{%
  \parbox{.333\textwidth}{#1}%
}

\title{
  \noindent
\makebox[0pt][l]{}%
\makebox[\textwidth][c]{AbcRanger}%
\makebox[0pt][r]{\raisebox{1.5ex}{\footnotesize{Code and binaries available at \url{http://github.com/diyabc/abcranger}\hspace{2cm}}}}\\
\LARGE{A fast and scalable random forest library for ABC model choice and parameter estimation}
}
% \title{\RaggedLeft \footnotesize{Code and binaries available at \url{http://github.com/diyabc/abcranger}}\\
% \Centering \Huge{AbcRanger}\\
%  \LARGE{A fast and scalable random forest library for ABC model choice and parameter estimation}}

\author{F.-D. Collin \inst{2} A. Estoup \inst{1} J.-M. Marin \inst{2} L. Raynal \inst{2} }

\institute[shortinst]{\inst{1} CBGP, INRA, CIRAD, IRD, Montpellier SupAgro, Univ. Montpellier \samelineand \inst{2} Université de Montpellier, CNRS, IMAG UMR 5149}

% \author{Alyssa P. Hacker \inst{1} \and Ben Bitdiddle \inst{2} \and Lem E. Tweakit \inst{2}}

% \institute[shortinst]{\inst{1} Some Institute \samelineand \inst{2} Another Institute}

% ====================
% Body
% ====================

\begin{document}

\begin{frame}[t]


\begin{columns}[t]
\separatorcolumn

\begin{column}{\colwidth}

  \begin{block}{First building block : ABC simulations}
    \begin{figure}
      \centering
        \includesvg[width=35cm]{abc-explication}
      \caption{ABC details}
    \end{figure}

    Given an observed data, the basic idea of ABC, \emph{Approximate Bayesian Computations} \cite{marin2012approximate}, is to approximate the likelihood of a parametrized model with selected simulations, by comparing the observed data and simulated ones via computed \emph{summary statistics}. The table of summary statistics for simulated data is called \realemph{the reference table}.
  
  \end{block}
  
  \begin{block}{ABC posterior methodologies}
    \begin{description}
      \item[\color{darkred}\sffamily \textbf{Model choice:}] Simulate data for several models and \emph{choose the best model to fit our data}
      \item[\color{darkred}\sffamily \textbf{Parameter estimation:}] Simulate data for one model and \emph{infer one or several parameters for this model given the observed data}
    \end{description}

    A sensible workflow is to first choose a model and then infer its parameters.

    \begin{figure}
      \centering
        \includesvg[width=35cm]{methodologies}
      \caption{ABC workflow with AbcRanger}
    \end{figure}

  \end{block}
  
  \begin{alertblock}{Challenges of ABC}
    in the context of population genetics recent advances
    \begin{description}
      \item[\color{darkyellow}\sffamily \textbf{Number of simulated data :}] could be > 100 000
      \item[\color{darkyellow}\sffamily \textbf{Number of summary statistics :}] could range from several hundred to tens of thousands (scenario with several populations and combinatory "explosion") : how to select the \emph{\color{darkred} meaningful} ones?
    \end{description}
    Classical Methods for ABC ($k$-nn and local methods) doesn't cope very well with this situation.
  \end{alertblock}
\begin{block}{Our solution}
  \cite{pudlo2015reliable} and \cite{raynal2016abc} proposed a novel approach, relying on \emph{Random Forests} to provide both model choice and parameter estimation methodologies
\end{block}
\end{column}

\separatorcolumn

\begin{column}{\colwidth}

  \begin{block}{Second building block : Random Forests}
    \begin{block}{CART}
      Random Forests are based on a CART, \emph{Classification and Regression Trees}, algorithm \cite{breiman:etal:1984}.

      \begin{wrapfigure}{l}{0.7\textwidth}
        \begin{center}
        \begin{tikzpicture}[scale=1.0, every node/.style={scale=1, minimum height=0.8cm}, thick]
            \useasboundingbox (-1,-1) rectangle (11.5,11);
            %\draw[very thin, gray] (0,0) grid[step=1] (10,10);
            
            \coordinate (c1) at (0,0);
            \coordinate (c2) at (0,10);
            \coordinate (c3) at (10,10);
            \coordinate (c4) at (10,0);
        
            \coordinate (X1) at (5,0);
            \coordinate (X2) at (0,5);
        
            \coordinate (cut1) at ($ (c1) + (0,4) $);
            \coordinate (cut1_bis) at ($ (c4) + (0,4) $);
        
            \coordinate (cut2) at ($ (c1) + (6,0) $);
            \coordinate (cut2_bis) at ($ (cut1) + (6,0) $);
        
            \coordinate (cut3) at ($ (cut1) + (3,0) $);
            \coordinate (cut3_bis) at ($ (c2) + (3,0) $);
            
            \coordinate (cut4) at ($ (cut3) + (0,4) $);
            \coordinate (cut4_bis) at ($ (cut1_bis) + (0,4) $);
        
            \coordinate (ret1) at (3,2);
            \coordinate (ret2) at (8,2);
            \coordinate (ret3) at (1.5,7);
            \coordinate (ret4) at (6.5,6);
            \coordinate (ret5) at (6.5,9);
        
            \coordinate (s1) at (0,4);
            \coordinate (s2) at (6,0);
            \coordinate (s3) at (3,0);
            \coordinate (s4) at (0,8);
        
            \draw (c1) -- (c2) -- (c3) -- (c4) -- (c1);
            \draw [color=lightblue,line width=1mm] (cut1) -- (cut1_bis);
            \draw [color=darkyellow,line width=1mm] (cut2) -- (cut2_bis);
            \draw [color=green!80!black,line width=1mm]  (cut3) -- (cut3_bis);
            \draw [color=red,line width=1mm] (cut4) -- (cut4_bis);
        
            \draw (ret1) node {$\hat{y}_1$};
            \draw (ret2) node {$\hat{y}_2$};
            \draw (ret3) node {$\hat{y}_3$};
            \draw (ret4) node {$\hat{y}_4$};
            \draw (ret5) node {$\hat{y}_5$};
        
            \draw [color=lightblue] (s1) node[left] {$s_1$};
            \draw [color=darkyellow] (s2) node[below] {$s_2$};
            \draw [color=green!80!black] (s3) node[below] {$s_3$};
            \draw [color=red] (s4) node[left] {$s_4$};
            
            \draw (s1) -- ++(-0.15,0);
            \draw (s2) -- ++(0,-0.15);
            \draw (s3) -- ++(0,-0.15);
            \draw (s4) -- ++(-0.15,0);
            
            \draw (X1) node[below, minimum height = 2.5cm] {$X_1$};
            \draw (X2) node[left, minimum width = 2.5cm] {$X_2$};
        
        \end{tikzpicture}%
        \begin{tikzpicture}[scale=1.5,
                edge from parent/.style={draw, thick, edge from parent path=
                {(\tikzparentnode.south)--+(0,-1.5pt)-| (\tikzchildnode)}},
                noeud/.style ={minimum size=1.2em, inner sep=4pt},
                level 1/.style={sibling distance=3cm},
                level 2/.style={sibling distance=2cm}]
                \tikzset{level distance=65pt}
                  \node [noeud,color=lightblue] {$X_2 \leq s_1$}
                    child {node [noeud,color=darkyellow] {$X_1 \leq s_2$} 
                        child {node [noeud] {$\hat{y}_1$} }
                        child {node [noeud] {$\hat{y}_2$} }
                    }
                    child {node [noeud,color=green!80!black] {$X_1 \leq s_3$} 
                        child {node [noeud] {$\hat{y}_3$} }
                        child {node [noeud,color=red] {$X_2 \leq s_4$} 
                            child {node [noeud] {$\hat{y}_4$} }
                            child {node [noeud] {$\hat{y}_5$} }
                        }
                    };
                \end{tikzpicture}
        \end{center}
        \caption{An example of CART and the associated partition of the two dimensional predictor space. Each splitting condition takes the form $X_j \leq s$ and the prediction at a leaf is denoted $\hat{y}_\ell$.}
        \label{chap2:fig:CART-tree}
        \end{wrapfigure}

        A CART is a \emph{machine learning algorithm} whose principle is to partition the predictor space into disjoint subspaces, in an iterative manner, and each one is assigned a prediction value which will be used for test data falling in this subspace.\\

        Once the partitioning is done, we have a binary tree structure which could predict outcomes from an input data, either classes or continuous values.

      \end{block}

    \begin{block}{Random Forests}

      \begin{figure}[ht]
        \centering
          \includesvg[width=35cm]{RF}
      \caption{Random Forest}
      \end{figure}

      \begin{multicols}{2}
        \textbf{Random Forests \cite{breiman:2001} are a three pronged extension of CART:}
        \begin{itemize}
          \item[\textbf{Ensemble method}] Training a \emph{set} of CART (not just one), and getting the majority vote (resp. mean) for classification (resp. regression)
          \item[\textbf{Bootstrapping}] Training data is random sampled (with replacement) for \emph{each} tree
          \item[\textbf{Feature bagging}] At each node of a growing tree, find the best split on a random subset of the features 
        \end{itemize}
        \columnbreak
         \centering
        \textbf{Advantages in an ABC setting : }
        \justify
        \begin{itemize}
          \item robust to noise
          \item (almost) free variable importance
          \item free (out-of-bag) cross-validation procedure
          \item easy parallelization
          \item good scaling properties on number of samples and on number of features (summary statistics)
          \item classifier and regressor
        \end{itemize}  
          
      \end{multicols}


    \end{block}
  \end{block}

  \begin{alertblock}{Computational challenges with ABC/Random Forests}

    With ~100 000 lines and more than 10 000 summary statistics, each tree could reach over 1 gigabyte of memory size. Typically we need 500 or 1000 trees for good prediction performance, so, even with state of the art RF packages like \cite{wright2015ranger}, memory constraints are preventing completion of the training.
    
  \end{alertblock}
  \begin{block}{A new implementation of Random Forest for ABC}
    Since ABC procedures only use trained Random Forests on a known set of observations, we have altered the random forest training computation by using only a subset of in-memory trees at a time and accumulating the required outcomes (predictions and statistics). Memory footprint is vastly improved and there is no performance cost.

    \begin{figure}[ht]
      \centering
      \includesvg[width=35cm]{RF-optim}
      \caption{Window of growing trees}
    \end{figure}

    \justifying Ongoing project \realemph{LeafLitter} intends to pursue that line even further: for a growing tree, only encountered leaves are stored. Thus, the memory footprint of trees becomes negligible, and could be parallelized at full scale.

  \end{block}

  

\end{column}

\separatorcolumn
\end{columns}



\begin{block}{\justifying References}
  \nocite{*}
  \footnotesize{\bibliographystyle{unsrt}\bibliography{poster}}

\end{block}



\end{frame}

\end{document}
