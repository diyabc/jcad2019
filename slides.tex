\documentclass{beamer}

\usepackage{svg}
\usepackage{xcolor}
\usepackage{tcolorbox}
\usepackage{pifont}
\usepackage{hyperref}
\usepackage{natbib}

\mode<presentation>

\title{ABCRanger}
\subtitle{A fast and scalable random forest library for ABC model choice and parameter estimation}
\author{F.-D. Collin \inst{2} A. Estoup \inst{1} J.-M. Marin \inst{2} L. Raynal \inst{2} }
\institute[shortinst]{\inst{1} CBGP, INRA, CIRAD, IRD, Montpellier SupAgro, Univ. Montpellier \and \inst{2} Université de Montpellier, CNRS, IMAG UMR 5149}
\date{\today}

\colorlet{darkyellow}{yellow!60!black}
\colorlet{lightblue}{blue!65}
\colorlet{darkred}{red!60!black}

\definecolor{mycolor}{rgb}{0.122, 0.435, 0.698}

\newtcbox{\mybox}{on line,
  colframe=mycolor,colback=mycolor!10!white,arc=4pt,boxsep=0pt,left=6pt,right=6pt,boxrule=0pt,top=6pt,bottom=6pt}

\newcommand{\realemph}[1]{\mybox{\textit{#1}}}

\begin{document}


    \begin{frame}
        \titlepage
    \end{frame}

    \begin{frame}
        \frametitle{Outline}
        \tableofcontents
    \end{frame}

    \section{Bayesian context}
    \subsection{Definitions and goal}
    \begin{frame}
        \frametitle{Approximate Bayesian Computation}
        
        It is defined  by :
        \begin{itemize}
            \item \emph{Bayesian Inference} context
            \item \emph{Likelihood-free inference} method
        \end{itemize}
            
    \end{frame}

    \subsection{Bayes Theorem}
    \begin{frame}
        \frametitle{Bayesian Inference}
        "If I believe in some model and I have some new data, what is the probability of this model, knowing this new data?".
        Baye's Theorem gives the answer :

        $$P(\Theta|Y) = \frac{P(Y|\Theta) * P(\Theta)}{P(Y)}$$

        Where :
        \begin{description}
            \item[Y] the \emph{data}, observations, evidence and so on. 
            \item[$\Theta$] the \emph{model} (hypothesis) we want to run
            \item[$P(\Theta)$] the \emph{prior probability} 
            \item[$P(\Theta|Y)$] the \emph{posterior probability} 
            \item[$P(Y|\Theta)$] the \emph{likelihood}
            \item[$P(Y)$] the \emph{marginal} likehood
        \end{description}

        
    \end{frame}

    \subsection{Introduction to ABC}
    \begin{frame}
        \frametitle{Likelihood free}
        \begin{itemize}
            \item $P(\Theta)$ is the easy part, $P(Y)$ should be too (and sometimes we can bypass it).
            \item Computing the \emph{likelihood} $P(Y|\Theta)$ is the name of the game.
        \end{itemize}
        
        Often we can't have a function for the likelihood, or it is intractable, too complex and so on.\\

        \ding{231} Enter the \emph{Likelihood-free} Kingdom and \emph{ABC (Approximate Bayesian Computation)}.

    \end{frame}

    \begin{frame}
        \frametitle{ABC in short}
        Given an observed data, the basic idea of ABC is to approximate the likelihood of a parametrized model with selected simulations, by comparing the observed data and simulated ones via computed \emph{summary statistics}.\\
        
        The table of summary statistics for simulated data is called \realemph{the reference table}.

    \end{frame}

    \begin{frame}
        \frametitle{ABC schema}

        \begin{figure}
            \centering
            \fontsize{6pt}{7.2}\selectfont
            \includesvg[height=8cm]{abc-explication}
            % \caption{ABC details}
        \end{figure}
    \end{frame}

    \section{Posterior Methodologies}

    \subsection{Workflow}
    \begin{frame}
        \frametitle{AbcRf/AbcRanger, presentation}
        \emph{AbcRanger} is a software for ABC posterior methodologies. It gets the output from an ABC run and provides :
        \begin{description}
            \item[\color{darkred}\sffamily \textbf{Model choice:}] Simulate data for several models and \emph{choose the best model to fit our data}
            \item[\color{darkred}\sffamily \textbf{Parameter estimation:}] Simulate data for one model and \emph{infer one or several parameters for this model given the observed data}
          \end{description}
    \end{frame}

    \begin{frame}
        \frametitle{ABC workflow with AbcRanger}

        \begin{figure}
            \centering
            \fontsize{6pt}{7.2}\selectfont

              \includesvg[height=8cm]{methodologies}
          \end{figure}
    \end{frame} 

    \subsection{Posterior methodologies}
    \begin{frame}
        \frametitle{Parameter estimation}
        Random Forest setup :
        \begin{itemize}
            \item Choose a parameter $t$ of the model 
            \item Train a regression RF on reftable with the $t$ as target
            \item Evaluate local/posteriors on observed data
        \end{itemize}
        \ding{231} Estimator for posterior PDF for the parameter (discretized but obtainable via kde) 
    \end{frame}

    \begin{frame}
        \frametitle{Model Choice}
        Two staged RF setup:
        \begin{enumerate}
            \item Classification :
                \begin{itemize}
                    \item Train a classification RF with the models (classes) as target
                    \item Eval the RF on observed data to get votes and chosen model
                
                \end{itemize}
            \item Regression : 
                \begin{itemize}
                    \item Using the previous RF, get the classified/misclassified on the training set as 0,1 and train a new regression RF with this as target
                    \item Evalute the obtained RF on the observed data to get the posterior probability of the chosen model
                \end{itemize}
        \end{enumerate}
    \end{frame}

    \subsection{Software and demos}
    \begin{frame}
        \frametitle{AbcRanger details}
        \begin{itemize}
            \item Written in C++, \url{http://github.com/diyabc/abcranger}, code and binaries (mac/windows/linux)
            \item Python frontend in the final stage (demos running)
            \item R frontend WIP
            \item optimized for large, high dimensional reference table without (too much) memory limits: more than $10^{e5}$ columns and $10^{e6}$ rows.
        \end{itemize}
    \end{frame}

    \begin{frame}
        \frametitle{Under the hood, a new RF implementation}
        Since ABC procedures only use trained Random Forests on a known set of observations, we have altered the random forest training computation by using only a subset of in-memory trees at a time and accumulating the required outcomes (predictions and statistics). Memory footprint is vastly improved and there is no performance cost.

        \begin{figure}
          \centering
          \fontsize{6pt}{7.2}\selectfont
          \includesvg[height=2.5cm]{RF-optim}
        \end{figure}
    
    \end{frame}

    \begin{frame}
        \frametitle{Demo time}
        \begin{itemize}
            \item \url{https://github.com/diyabc/abcranger/blob/master/testpy/Parameter\%20Estimation\%20Demo.ipynb}
            \item \url{https://github.com/diyabc/abcranger/blob/master/testpy/Model\%20Choice\%20Demo.ipynb}
        \end{itemize}
    \end{frame}

    \section{Thoughts and Perspectives}
    \begin{frame}
        \frametitle{Conclusion}
        \begin{enumerate}
            \item Thoughts:
                \begin{itemize}
                    \item nice integration of ML techniques in a model-based approach...
                    \item ... although the objective there is not better "predictions" or "score" as in ML but easy and accurate posteriors
                \end{itemize}
            \item Perspectives:
                \begin{itemize}
                    \item deeper integration in ABC pipeline like the Elfi python package
                    \item On the RF side, ongoing project \realemph{LeafLitter} intends to pursue that line even further: for a growing tree, only encountered leaves are stored. Thus, the memory footprint of the trees becomes negligible, and their growing could finally be parallelized at full scale.
                \end{itemize}
        \end{enumerate}
    \end{frame}

    \section{Reference}
    \begin{frame}{allowframebreaks}
        \frametitle{References}
        \nocite{*}
        \fontsize{10pt}{7.2}\selectfont

        \bibliographystyle{unsrt}\bibliography{poster}
    \end{frame}

\end{document}

